%%=============================================================================
%% Inleiding
%%=============================================================================

\chapter{\IfLanguageName{dutch}{Inleiding}{Introduction}}%
\label{ch:inleiding}

De Schelde is een belangrijke waterweg die door meerdere landen stroomt en een cruciale rol speelt in de economie, natuur en leefbaarheid van de omgeving. Het verzamelen en delen van gegevens over menselijke activiteiten in de Schelde is van groot belang voor het monitoren en beheren van deze waterweg. Echter, het gebruik van verschillende gegevensformaten door diverse organisaties vormt een uitdaging voor het vergelijken en analyseren van deze gegevens.

\section{\IfLanguageName{dutch}{Probleemstelling}{Problem Statement}}%
\label{sec:probleemstelling}

Het gebruik van verschillende gegevensformaten door diverse organisaties bemoeilijkt het vergelijken en analyseren van gegevens over menselijke activiteiten in de Schelde. Deze verscheidenheid aan formaten leidt tot inconsistenties en beperkingen bij het verkrijgen van een geïntegreerd beeld van de activiteiten in de waterweg. Het ontbreken van een uniforme gegevensindeling belemmert effectieve monitoring, samenwerking en besluitvorming met betrekking tot het beheer van de Schelde. Het is daarom van essentieel belang om een gestandaardiseerd formaat te ontwikkelen dat de vergelijkbaarheid en analyse van gegevens verbetert, zodat de Schelde optimaal beheerd kan worden als een cruciale waterweg voor economie, natuur en leefbaarheid. Deze thesis richt zich op het identificeren van de uitdagingen die worden veroorzaakt door het gebruik van verschillende gegevensformaten en streeft naar aanbevelingen voor een uniforme gegevensindeling voor informatie over menselijke activiteiten in de Schelde.

\section{\IfLanguageName{dutch}{Onderzoeksvraag}{Research question}}%
\label{sec:onderzoeksvraag}

\subsection{Hoofdonderzoeksvraag}
Wat zijn de essentiële elementen en de optimale structuur van een uniform dataformaat voor gegevens over menselijke activiteiten in de Schelde, met als doel de vergelijkbaarheid en analyse van deze gegevens te verbeteren?

Deze onderzoeksvraag richt zich op het identificeren van de belangrijkste elementen en de meest geschikte structuur die nodig zijn voor het ontwikkelen van een uniform dataformaat. Het beoogde resultaat van dit onderzoek is het voorstellen van een gestandaardiseerd dataformaat dat de vergelijkbaarheid en analyse van gegevens over menselijke activiteiten in de Schelde optimaliseert.

\subsection{Deelonderzoeksvraag}
Welke bekende en minder bekende bestandsformaten worden gebruikt voor het opslaan en delen van gegevens over menselijke activiteiten, en wat zijn de belangrijkste kenmerken en voor- en nadelen van deze bestandsformaten?

Deze deelonderzoeksvraag richt zich op het verkennen van verschillende bestandsformaten die worden gebruikt voor het opslaan en delen van gegevens over menselijke activiteiten. Het doel is om een overzicht te krijgen van zowel bekende als minder bekende bestandsformaten en hun kenmerken, voor- en nadelen te identificeren. Het resultaat van dit deelonderzoek zal een vergelijkende studie zijn die professionals en beleidsmakers inzicht geeft in de verschillende bestandsformaten die beschikbaar zijn en hun relevantie voor het onderzoek naar menselijke activiteiten in de Schelde.

\section{\IfLanguageName{dutch}{Onderzoeksdoelstelling}{Research objective}}%
\label{sec:onderzoeksdoelstelling}

Het hoofddoel van dit onderzoek is het identificeren en formuleren van een gestandaardiseerd dataformaat voor gegevens over menselijke activiteiten in de Schelde. Dit omvat het onderzoeken van bestaande dataportalen en standaard gegevensformaten, het analyseren van de specifieke behoeften en uitdagingen binnen de context van de Schelde en het ontwikkelen van aanbevelingen voor een uniforme en consistente structuur van de gegevens. De onderzoeksdoelstelling is gericht op het verbeteren van de vergelijkbaarheid, analyse en samenwerking op basis van gestandaardiseerde gegevensindelingen, en het faciliteren van effectief beheer en behoud van de Schelde als belangrijke waterweg.

\section{\IfLanguageName{dutch}{Aanbevelingen en beperkingen}{recomendations objective}}%
\label{sec:Aanbevelingen-beperkingen}

Na een grondige analyse van de verzamelde informatie en de resultaten van het onderzoek, wordt een voorgesteld uniform bestandsformaat gepresenteerd. Dit bestandsformaat is ontwikkeld met het oog op het vereenvoudigen van gegevensuitwisseling en het bevorderen van de samenwerking tussen verschillende organisaties.

Het uniforme bestandsformaat omvat duidelijke richtlijnen voor het opmaken en structureren van gegevens over menselijke activiteiten in de Schelde. Het legt de nadruk op gestandaardiseerde terminologieën en essentiële gegevenselementen, waardoor de vergelijkbaarheid en analyse van gegevens worden verbeterd.

Het is echter belangrijk om enkele beperkingen van het voorgestelde uniforme bestandsformaat te benadrukken. Hoewel het zorgvuldig is ontworpen om aan de behoeften van het onderzoek te voldoen, kan het mogelijk niet volledig alle specifieke eisen van individuele organisaties en toekomstige ontwikkelingen binnen het domein omvatten. Daarom is het belangrijk om flexibiliteit en aanpassingsvermogen te behouden bij het implementeren van dit bestandsformaat.

Door het implementeren van de voorgestelde aanbevelingen en het erkennen van de beperkingen kunnen organisaties een belangrijke stap zetten in het harmoniseren van gegevensindelingen en het verbeteren van de samenwerking op het gebied van gegevensverzameling en -analyse.

\section{\IfLanguageName{dutch}{Opzet van deze bachelorproef}{Structure of this bachelor thesis}}%
\label{sec:opzet-bachelorproef}

% Het is gebruikelijk aan het einde van de inleiding een overzicht te
% geven van de opbouw van de rest van de tekst. Deze sectie bevat al een aanzet
% die je kan aanvullen/aanpassen in functie van je eigen tekst.

De rest van deze bachelorproef is als volgt opgebouwd:

In Hoofdstuk~\ref{ch:stand-van-zaken} wordt een overzicht gegeven van de stand van zaken binnen het onderzoeksdomein, op basis van een literatuurstudie.

In Hoofdstuk~\ref{ch:methodologie} wordt de methodologie toegelicht en worden de gebruikte onderzoekstechnieken besproken om een antwoord te kunnen formuleren op de onderzoeksvragen.

In Hoofdstuk~\ref{ch:data} wordt een overzicht gegeven van de gevonden datastructuren.

In Hoofdstuk~\ref{ch:standaard-formaat} wordt toegelicht hoe de uniforme structuur er uitziet en waarom.

In Hoofdstuk~\ref{ch:conclusie}, tenslotte, wordt de conclusie gegeven en een antwoord geformuleerd op de onderzoeksvragen. Daarbij wordt ook een aanzet gegeven voor toekomstig onderzoek binnen dit domein.