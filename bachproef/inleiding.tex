%%=============================================================================
%% Inleiding
%%=============================================================================

\chapter{\IfLanguageName{dutch}{Inleiding}{Introduction}}%
\label{ch:inleiding}

De Schelde is een belangrijke waterweg in België en Nederland, waarlangs menselijke activiteiten plaatsvinden zoals scheepvaart, visserij, recreatie en industrie. Het verzamelen van gegevens over deze activiteiten is van groot belang voor het monitoren en beheren van de Schelde en het inschatten van de effecten op het milieu en de samenleving. Om deze gegevens goed te kunnen analyseren en vergelijken, is uniformiteit in de gebruikte dataformaten essentieel. Dit onderzoek richt zich dan ook op het ontwikkelen van uniforme dataformaten voor gegevens over menselijke activiteiten in de Schelde.

Het probleem is echter dat er momenteel geen uniformiteit bestaat in de dataformaten die worden gebruikt door verschillende instanties die gegevens verzamelen over menselijke activiteiten in de Schelde. Dit kan leiden tot inconsistenties en onduidelijkheid bij het vergelijken van de gegevens en het uitvoeren van analyses. Daarom is er behoefte aan uniformiteit in de gebruikte dataformaten.

Het doel van dit onderzoek is dan ook om aanbevelingen te formuleren voor het verbeteren van de uniformiteit van deze dataformaten. Dit onderzoek is uitgevoerd in opdracht van de Hogeschool Gent, met Chantal Teerlinck als promotor en Charlotte Dhondt als co-promotor vanuit het Vlaams Instituut voor de Zee (VLIZ).

Deze inleiding geeft een overzicht van de context en de noodzaak van het onderzoek. In het volgende deel van de inleiding zal de onderzoeksvraag worden geformuleerd en zal de methodologie van het onderzoek worden besproken.

\section{\IfLanguageName{dutch}{Probleemstelling}{Problem Statement}}%
\label{sec:probleemstelling}

De problematiek die aan de basis ligt van dit onderzoek is de nood aan uniforme dataformaten voor gegevens over menselijke activiteiten in de Schelde. Momenteel worden er verschillende dataformaten gebruikt door verschillende instanties die data verzamelen over de activiteiten die plaatsvinden op en rond de Schelde. Dit leidt tot een grote verscheidenheid aan dataformaten, waardoor het moeilijk is om deze data te vergelijken en te integreren. Bovendien zorgt dit voor een versnippering van de informatie, waardoor het moeilijk wordt om een holistisch beeld te vormen van de activiteiten in de Schelde.

Deze problematiek heeft gevolgen voor verschillende doelgroepen die geïnteresseerd zijn in de menselijke activiteiten in de Schelde. Zo hebben bijvoorbeeld wetenschappers nood aan deze data om onderzoek te doen naar de impact van de activiteiten op het milieu en de biodiversiteit van de Schelde. Daarnaast hebben ook beleidsmakers, waterbeheerders en havenautoriteiten nood aan een holistisch beeld van de activiteiten in de Schelde om beslissingen te kunnen nemen op vlak van beleid en beheer.

Het doel van dit onderzoek is dan ook om een uniform dataformaat te ontwikkelen dat door alle betrokken partijen gebruikt kan worden om data te verzamelen over menselijke activiteiten in de Schelde. Dit zal ervoor zorgen dat de informatie eenduidig en vergelijkbaar is, waardoor het mogelijk wordt om een holistisch beeld te vormen van de activiteiten in de Schelde.

\section{\IfLanguageName{dutch}{Onderzoeksvraag}{Research question}}%
\label{sec:onderzoeksvraag}

De centrale onderzoeksvraag die in deze bachelorproef wordt beantwoord luidt als volgt: "Hoe kunnen uniforme dataformaten voor gegevens over menselijke activiteiten in de Schelde worden ontwikkeld, en wat zijn de aanbevelingen om de uniformiteit van deze dataformaten te verbeteren?"

\section{\IfLanguageName{dutch}{Onderzoeksdoelstelling}{Research objective}}%
\label{sec:onderzoeksdoelstelling}

Het doel van deze bachelorproef is om een overzicht te bieden van uniforme dataformaten voor het meten en verzamelen van gegevens over menselijke activiteiten in de Schelde, en om aanbevelingen te doen voor het gebruik van deze dataformaten. De succescriteria voor deze bachelorproef zijn:

1. Een grondige analyse van de verschillende dataformaten die momenteel beschikbaar zijn voor het meten en verzamelen van gegevens over menselijke activiteiten in de Schelde.

2. Een vergelijkende studie van de voor- en nadelen van deze dataformaten.

3. Een set van aanbevelingen voor het gebruik van uniforme dataformaten in de praktijk, gebaseerd op de resultaten van de analyse en de vergelijkende studie.

Het beoogde resultaat van deze bachelorproef is een verslag met concrete aanbevelingen voor het gebruik van uniforme dataformaten bij het meten en verzamelen van gegevens over menselijke activiteiten in de Schelde.

\section{\IfLanguageName{dutch}{Conclusies, aanbevelingen en beperkingen}{recomendations objective}}%
\label{sec:Conclusies-aanbevelingen-beperkingen}

Deze resultaten hebben implicaties voor het werkveld, omdat de uniforme dataformaten kunnen bijdragen aan een betere analyse en vergelijking van gegevens over menselijke activiteiten in de Schelde. Echter, er zijn ook beperkingen aan dit onderzoek, zoals de beperkte tijd en middelen om alle relevante instanties te interviewen en de beperkingen in de beschikbare literatuur.

In deze bachelorproef worden de resultaten en aanbevelingen verder uitgewerkt en toegelicht.

\section{\IfLanguageName{dutch}{Opzet van deze bachelorproef}{Structure of this bachelor thesis}}%
\label{sec:opzet-bachelorproef}

% Het is gebruikelijk aan het einde van de inleiding een overzicht te
% geven van de opbouw van de rest van de tekst. Deze sectie bevat al een aanzet
% die je kan aanvullen/aanpassen in functie van je eigen tekst.

De rest van deze bachelorproef is als volgt opgebouwd:

In Hoofdstuk~\ref{ch:stand-van-zaken} wordt een overzicht gegeven van de stand van zaken binnen het onderzoeksdomein, op basis van een literatuurstudie.

In Hoofdstuk~\ref{ch:methodologie} wordt de methodologie toegelicht en worden de gebruikte onderzoekstechnieken besproken om een antwoord te kunnen formuleren op de onderzoeksvragen.

% TODO: Vul hier aan voor je eigen hoofstukken, één of twee zinnen per hoofdstuk

In Hoofdstuk~\ref{ch:conclusie}, tenslotte, wordt de conclusie gegeven en een antwoord geformuleerd op de onderzoeksvragen. Daarbij wordt ook een aanzet gegeven voor toekomstig onderzoek binnen dit domein.