%%=============================================================================
%% Methodologie
%%=============================================================================

\chapter{\IfLanguageName{dutch}{Methodologie}{Methodology}}%
\label{ch:methodologie}

\textbf{Deel 1: Literatuurstudie over FAIR- en open-standaarden}

1.1 Definitie en belang van FAIR- en open-standaarden

Onderzoeken wat FAIR- en open-standaarden zijn en waarom ze belangrijk zijn in het kader van het onderzoek.
Literatuurstudie over FAIR-principes en open standaarden, zoals de Open Geospatial Consortium (OGC), ISO/IEC, en W3C-standaarden.

1.2 Ontstaan en evolutie van FAIR- en open-standaarden

Onderzoeken hoe FAIR- en open-standaarden zijn ontstaan en zich hebben ontwikkeld tot wat ze vandaag zijn.
Literatuurstudie over de geschiedenis van FAIR- en open-standaarden, zoals de ontwikkeling van de OGC-standaarden en de ISO/IEC-normen.

1.3 Analyse van bestandsformaten

Op basis van de literatuurstudie, bepalen welke bestandsformaten in aanmerking komen voor het eindresultaat.
Beoordelen welke bestandsformaten voldoen aan de FAIR-principes en open-standaarden.

\textbf{Deel 2: Interviews met betrokken personen}

2.1 Interviews met verstrekkers van baggergegevens

Interviews afnemen met de huidige verstrekkers van baggergegevens om te achterhalen welke bestandsformaten zij momenteel gebruiken en waarom.
Vragen stellen over eventuele problemen met de huidige bestandsformaten.

2.2 Interviews met opdrachtgevers en wetenschappelijke experts

Interviews afnemen met de betrokken opdrachtgevers en wetenschappelijke experts om te achterhalen welke bestandsformaten zij voorkeur voor hebben en waarom.
Vergelijken van verschillende dataformaten.

\textbf{Deel 3: Implementatie van de data in het datasysteem of proof-of-concept}

3.1 Implementatie in datasysteem

Op basis van de resultaten uit de literatuurstudie en de interviews, implementeren van de uniforme data in het datasysteem.
Overleggen met de verantwoordelijke van het datasysteem om te bepalen of de implementatie effectief is en er geen problemen ontstaan.

3.2 Proof-of-concept

Indien de implementatie in het datasysteem niet mogelijk is, zal een proof-of-concept worden ontwikkeld.
Evalueren of de uniforme data effectief kan worden opgenomen in het proof-of-concept.

%% TODO: Hoe ben je te werk gegaan? Verdeel je onderzoek in grote fasen, en
%% licht in elke fase toe welke stappen je gevolgd hebt. Verantwoord waarom je
%% op deze manier te werk gegaan bent. Je moet kunnen aantonen dat je de best
%% mogelijke manier toegepast hebt om een antwoord te vinden op de
%% onderzoeksvraag.

\lipsum[21-25]

