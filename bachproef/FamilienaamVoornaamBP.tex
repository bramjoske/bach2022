%===============================================================================
% LaTeX sjabloon voor de bachelorproef toegepaste informatica aan HOGENT
% Meer info op https://github.com/HoGentTIN/latex-hogent-report
%===============================================================================

\documentclass[dutch,dit,thesis]{hogentreport}

% TODO:
% - If necessary, replace the option `dit`' with your own department!
%   Valid entries are dbo, dbt, dgz, dit, dlo, dog, dsa, soa
% - If you write your thesis in English (remark: only possible after getting
%   explicit approval!), remove the option "dutch," or replace with "english".

\usepackage{lipsum} % For blind text, can be removed after adding actual content

%% Pictures to include in the text can be put in the graphics/ folder
\graphicspath{{graphics/}}

%% For source code highlighting, requires pygments to be installed
%% Compile with the -shell-escape flag!
\usepackage[section]{minted}
\usemintedstyle{solarized-light}
\definecolor{bg}{RGB}{253,246,227} %% Set the background color of the codeframe

%% Change this line to edit the line numbering style:
\renewcommand{\theFancyVerbLine}{\ttfamily\scriptsize\arabic{FancyVerbLine}}

%% Macro definition to load external java source files with \javacode{filename}:
\newmintedfile[javacode]{java}{
    bgcolor=bg,
    fontfamily=tt,
    linenos=true,
    numberblanklines=true,
    numbersep=5pt,
    gobble=0,
    framesep=2mm,
    funcnamehighlighting=true,
    tabsize=4,
    obeytabs=false,
    breaklines=true,
    mathescape=false
    samepage=false,
    showspaces=false,
    showtabs =false,
    texcl=false,
}

% Other packages not already included can be imported here

%%---------- Document metadata -------------------------------------------------
% TODO: Replace this with your own information
\author{Bram Verbanck}
\supervisor{Mevr. C. Teerlinck}
\cosupervisor{Mevr. C. Dhondt}
\title[Optionele ondertitel]%
    {Uniforme bestandsformaten voor gegevens over menselijke activiteiten in de schelde.}
\academicyear{\advance\year by -1 \the\year--\advance\year by 1 \the\year}
\examperiod{1}
\degreesought{\IfLanguageName{dutch}{Professionele bachelor in de toegepaste informatica}{Bachelor of applied computer science}}
\partialthesis{false} %% To display 'in partial fulfilment'
%\institution{Internshipcompany BVBA.}

%% Add global exceptions to the hyphenation here
\hyphenation{back-slash}

%% The bibliography (style and settings are  found in hogentthesis.cls)
\addbibresource{bachproef.bib}            %% Bibliography file
\addbibresource{../voorstel/voorstel.bib} %% Bibliography research proposal
\defbibheading{bibempty}{}

%% Prevent empty pages for right-handed chapter starts in twoside mode
\renewcommand{\cleardoublepage}{\clearpage}

\renewcommand{\arraystretch}{1.2}

%% Content starts here.
\begin{document}

%---------- Front matter -------------------------------------------------------

\frontmatter

\hypersetup{pageanchor=false} %% Disable page numbering references
%% Render a Dutch outer title page if the main language is English
\IfLanguageName{english}{%
    %% If necessary, information can be changed here
    \degreesought{Professionele Bachelor toegepaste informatica}%
    \begin{otherlanguage}{dutch}%
       \maketitle%
    \end{otherlanguage}%
}{}

%% Generates title page content
\maketitle
\hypersetup{pageanchor=true}

%%=============================================================================
%% Voorwoord
%%=============================================================================

\chapter*{\IfLanguageName{dutch}{Woord vooraf}{Preface}}%
\label{ch:voorwoord}

%% TODO:
%% Het voorwoord is het enige deel van de bachelorproef waar je vanuit je
%% eigen standpunt (``ik-vorm'') mag schrijven. Je kan hier bv. motiveren
%% waarom jij het onderwerp wil bespreken.
%% Vergeet ook niet te bedanken wie je geholpen/gesteund/... heeft

Beste lezers,

Met trots presenteer ik mijn bachelorscriptie over uniforme dataformaten voor gegevens over menselijke activiteiten in de Schelde. Als student aan de Hogeschool Gent heb ik de afgelopen maanden intensief samengewerkt met mijn promotor, Chantal Teerlinck, en mijn co-promotor, Charlotte Dhondt van het Vlaams Instituut voor de Zee (VLIZ). Zij hebben mij gedurende het hele proces begeleid en ondersteund in mijn onderzoek.

De Schelde is een belangrijke waterweg die door meerdere landen stroomt en een cruciale rol speelt in de economie, natuur en leefbaarheid van de omgeving. Het verzamelen en delen van gegevens over menselijke activiteiten in de Schelde is van groot belang voor het monitoren en beheren van deze waterweg. Echter, het gebruik van verschillende dataformaten door verschillende instanties maakt het lastig om deze gegevens goed te vergelijken en te analyseren.

In mijn scriptie heb ik me gericht op het ontwikkelen van uniforme dataformaten voor gegevens over menselijke activiteiten in de Schelde. Ik heb onderzocht welke factoren van invloed zijn op de uniformiteit van deze dataformaten en hoe deze factoren kunnen worden aangepakt. Hierbij heb ik gebruik gemaakt van verschillende bronnen, waaronder literatuuronderzoek en interviews met experts.

Het resultaat van mijn onderzoek is een set van aanbevelingen voor uniforme dataformaten voor gegevens over menselijke activiteiten in de Schelde. Deze aanbevelingen kunnen worden gebruikt door instanties die zich bezighouden met het verzamelen en delen van deze gegevens, om zo de uniformiteit en vergelijkbaarheid van de data te verbeteren.

Ik hoop dat mijn scriptie bijdraagt aan een beter begrip van het belang van uniforme dataformaten voor gegevens over menselijke activiteiten in de Schelde, en dat de aanbevelingen die ik heb geformuleerd worden gebruikt om de samenwerking tussen verschillende instanties te verbeteren.

Met vriendelijke groet,

Bram Verbanck
%%=============================================================================
%% Samenvatting
%%=============================================================================

% TODO: De "abstract" of samenvatting is een kernachtige (~ 1 blz. voor een
% thesis) synthese van het document.
%
% Een goede abstract biedt een kernachtig antwoord op volgende vragen:
%
% 1. Waarover gaat de bachelorproef?
% 2. Waarom heb je er over geschreven?
% 3. Hoe heb je het onderzoek uitgevoerd?
% 4. Wat waren de resultaten? Wat blijkt uit je onderzoek?
% 5. Wat betekenen je resultaten? Wat is de relevantie voor het werkveld?
%
% Daarom bestaat een abstract uit volgende componenten:
%
% - inleiding + kaderen thema
% - probleemstelling
% - (centrale) onderzoeksvraag
% - onderzoeksdoelstelling
% - methodologie
% - resultaten (beperk tot de belangrijkste, relevant voor de onderzoeksvraag)
% - conclusies, aanbevelingen, beperkingen
%
% LET OP! Een samenvatting is GEEN voorwoord!

%%---------- Nederlandse samenvatting -----------------------------------------
%
% TODO: Als je je bachelorproef in het Engels schrijft, moet je eerst een
% Nederlandse samenvatting invoegen. Haal daarvoor onderstaande code uit
% commentaar.
% Wie zijn bachelorproef in het Nederlands schrijft, kan dit negeren, de inhoud
% wordt niet in het document ingevoegd.


%%---------- Samenvatting -----------------------------------------------------
% De samenvatting in de hoofdtaal van het document

\chapter*{\IfLanguageName{dutch}{Samenvatting}{Abstract}}

Deze scriptie presenteert een onderzoek naar gestandaardiseerde gegevensindelingen voor informatie over menselijke activiteiten in de Schelde. Het doel van dit onderzoek is het verbeteren van de vergelijkbaarheid en analyse van deze gegevens, die van cruciaal belang zijn voor het monitoren en beheren van deze belangrijke waterweg.

De probleemstelling gaat over de uitdagingen door het gebruik van verschillende gegevensindelingen door diverse organisaties. Dit bemoeilijkt het vergelijken en analyseren van de gegevens, wat leidt tot beperkte inzichten en samenwerking.

Om deze problemen aan te pakken, is een grondige aanpak gevolgd. Allereerst is uitgebreid literatuuronderzoek uitgevoerd om bestaande benaderingen en standaarden voor gegevensindelingen te begrijpen. Daarnaast zijn er gesprekken gevoerd met belanghebbenden om de specifieke behoeften en uitdagingen binnen de context van de Schelde te identificeren.

Als resultaat is er een aanbevolen gegevensindeling ontstaan, gebaseerd op de verzamelde informatie. Deze indeling richt zich op het identificeren van essentiële gegevenselementen, het vaststellen van gestandaardiseerde terminologieën en het definiëren van een consistente structuur voor de gegevens.

De resultaten van dit onderzoek bieden waardevolle inzichten voor organisaties die verantwoordelijk zijn voor het verzamelen en delen van informatie over menselijke activiteiten in de Schelde. Door de voorgestelde gestandaardiseerde gegevensindelingen te implementeren, kunnen zij de uniformiteit en vergelijkbaarheid van de gegevens verbeteren, wat resulteert in betere besluitvorming, samenwerking en beheer van de waterweg.

De belangrijkste conclusie van dit onderzoek is dat het gebruik van gestandaardiseerde gegevensindelingen een positieve impact kan hebben op de efficiëntie en effectiviteit van het beheer van de Schelde. Het bevordert de samenwerking tussen verschillende organisaties en maakt een diepgaandere analyse en interpretatie van de gegevens mogelijk.

De persoonlijke bijdrage aan dit onderzoek omvat het grondig bestuderen van de literatuur, actieve deelname aan gesprekken met belanghebbenden en het formuleren van concrete aanbevelingen. Het onderzoek draagt bij aan het begrip van het belang van gestandaardiseerde gegevensindelingen voor informatie over menselijke activiteiten in de Schelde en de praktische toepassing ervan in de realiteit.



%---------- Inhoud, lijst figuren, ... -----------------------------------------

\tableofcontents

% In a list of figures, the complete caption will be included. To prevent this,
% ALWAYS add a short description in the caption!
%
%  \caption[short description]{elaborate description}
%
% If you do, only the short description will be used in the list of figures

\listoffigures

% If you included tables and/or source code listings, uncomment the appropriate
% lines.
%\listoftables
%\listoflistings

% Als je een lijst van afkortingen of termen wil toevoegen, dan hoort die
% hier thuis. Gebruik bijvoorbeeld de ``glossaries'' package.
% https://www.overleaf.com/learn/latex/Glossaries

%---------- Kern ---------------------------------------------------------------

\mainmatter{}

% De eerste hoofdstukken van een bachelorproef zijn meestal een inleiding op
% het onderwerp, literatuurstudie en verantwoording methodologie.
% Aarzel niet om een meer beschrijvende titel aan deze hoofdstukken te geven of
% om bijvoorbeeld de inleiding en/of stand van zaken over meerdere hoofdstukken
% te verspreiden!

%%=============================================================================
%% Inleiding
%%=============================================================================

\chapter{\IfLanguageName{dutch}{Inleiding}{Introduction}}%
\label{ch:inleiding}

De Schelde is een belangrijke waterweg die door meerdere landen stroomt en een cruciale rol speelt in de economie, natuur en leefbaarheid van de omgeving. Het verzamelen en delen van gegevens over menselijke activiteiten in de Schelde is van groot belang voor het monitoren en beheren van deze waterweg. Echter, het gebruik van verschillende gegevensformaten door diverse organisaties vormt een uitdaging voor het vergelijken en analyseren van deze gegevens.

\section{\IfLanguageName{dutch}{Probleemstelling}{Problem Statement}}%
\label{sec:probleemstelling}

Het gebruik van verschillende gegevensformaten door diverse organisaties bemoeilijkt het vergelijken en analyseren van gegevens over menselijke activiteiten in de Schelde. Deze verscheidenheid aan formaten leidt tot inconsistenties en beperkingen bij het verkrijgen van een geïntegreerd beeld van de activiteiten in de waterweg. Het ontbreken van een uniforme gegevensindeling belemmert effectieve monitoring, samenwerking en besluitvorming met betrekking tot het beheer van de Schelde. Het is daarom van essentieel belang om een gestandaardiseerd formaat te ontwikkelen dat de vergelijkbaarheid en analyse van gegevens verbetert, zodat de Schelde optimaal beheerd kan worden als een cruciale waterweg voor economie, natuur en leefbaarheid. Deze thesis richt zich op het identificeren van de uitdagingen die worden veroorzaakt door het gebruik van verschillende gegevensformaten en streeft naar aanbevelingen voor een uniforme gegevensindeling voor informatie over menselijke activiteiten in de Schelde.

\section{\IfLanguageName{dutch}{Onderzoeksvraag}{Research question}}%
\label{sec:onderzoeksvraag}

\subsection{Hoofdonderzoeksvraag}
Wat zijn de essentiële elementen en de optimale structuur van een uniform dataformaat voor gegevens over menselijke activiteiten in de Schelde, met als doel de vergelijkbaarheid en analyse van deze gegevens te verbeteren?

Deze onderzoeksvraag richt zich op het identificeren van de belangrijkste elementen en de meest geschikte structuur die nodig zijn voor het ontwikkelen van een uniform dataformaat. Het beoogde resultaat van dit onderzoek is het voorstellen van een gestandaardiseerd dataformaat dat de vergelijkbaarheid en analyse van gegevens over menselijke activiteiten in de Schelde optimaliseert.

\subsection{Deelonderzoeksvraag}
Welke bekende en minder bekende bestandsformaten worden gebruikt voor het opslaan en delen van gegevens over menselijke activiteiten, en wat zijn de belangrijkste kenmerken en voor- en nadelen van deze bestandsformaten?

Deze deelonderzoeksvraag richt zich op het verkennen van verschillende bestandsformaten die worden gebruikt voor het opslaan en delen van gegevens over menselijke activiteiten. Het doel is om een overzicht te krijgen van zowel bekende als minder bekende bestandsformaten en hun kenmerken, voor- en nadelen te identificeren. Het resultaat van dit deelonderzoek zal een vergelijkende studie zijn die professionals en beleidsmakers inzicht geeft in de verschillende bestandsformaten die beschikbaar zijn en hun relevantie voor het onderzoek naar menselijke activiteiten in de Schelde.

\section{\IfLanguageName{dutch}{Onderzoeksdoelstelling}{Research objective}}%
\label{sec:onderzoeksdoelstelling}

Het hoofddoel van dit onderzoek is het identificeren en formuleren van een gestandaardiseerd dataformaat voor gegevens over menselijke activiteiten in de Schelde. Dit omvat het onderzoeken van bestaande dataportalen en standaard gegevensformaten, het analyseren van de specifieke behoeften en uitdagingen binnen de context van de Schelde en het ontwikkelen van aanbevelingen voor een uniforme en consistente structuur van de gegevens. De onderzoeksdoelstelling is gericht op het verbeteren van de vergelijkbaarheid, analyse en samenwerking op basis van gestandaardiseerde gegevensindelingen, en het faciliteren van effectief beheer en behoud van de Schelde als belangrijke waterweg.

\section{\IfLanguageName{dutch}{Aanbevelingen en beperkingen}{recomendations objective}}%
\label{sec:Aanbevelingen-beperkingen}

Na een grondige analyse van de verzamelde informatie en de resultaten van het onderzoek, wordt een voorgesteld uniform bestandsformaat gepresenteerd. Dit bestandsformaat is ontwikkeld met het oog op het vereenvoudigen van gegevensuitwisseling en het bevorderen van de samenwerking tussen verschillende organisaties.

Het uniforme bestandsformaat omvat duidelijke richtlijnen voor het opmaken en structureren van gegevens over menselijke activiteiten in de Schelde. Het legt de nadruk op gestandaardiseerde terminologieën en essentiële gegevenselementen, waardoor de vergelijkbaarheid en analyse van gegevens worden verbeterd.

Het is echter belangrijk om enkele beperkingen van het voorgestelde uniforme bestandsformaat te benadrukken. Hoewel het zorgvuldig is ontworpen om aan de behoeften van het onderzoek te voldoen, kan het mogelijk niet volledig alle specifieke eisen van individuele organisaties en toekomstige ontwikkelingen binnen het domein omvatten. Daarom is het belangrijk om flexibiliteit en aanpassingsvermogen te behouden bij het implementeren van dit bestandsformaat.

Door het implementeren van de voorgestelde aanbevelingen en het erkennen van de beperkingen kunnen organisaties een belangrijke stap zetten in het harmoniseren van gegevensindelingen en het verbeteren van de samenwerking op het gebied van gegevensverzameling en -analyse.

\section{\IfLanguageName{dutch}{Opzet van deze bachelorproef}{Structure of this bachelor thesis}}%
\label{sec:opzet-bachelorproef}

% Het is gebruikelijk aan het einde van de inleiding een overzicht te
% geven van de opbouw van de rest van de tekst. Deze sectie bevat al een aanzet
% die je kan aanvullen/aanpassen in functie van je eigen tekst.

De rest van deze bachelorproef is als volgt opgebouwd:

In Hoofdstuk~\ref{ch:stand-van-zaken} wordt een overzicht gegeven van de stand van zaken binnen het onderzoeksdomein, op basis van een literatuurstudie.

In Hoofdstuk~\ref{ch:methodologie} wordt de methodologie toegelicht en worden de gebruikte onderzoekstechnieken besproken om een antwoord te kunnen formuleren op de onderzoeksvragen.

In Hoofdstuk~\ref{ch:data} wordt een overzicht gegeven van de gevonden datastructuren.

In Hoofdstuk~\ref{ch:standaard-formaat} wordt toegelicht hoe de uniforme structuur er uitziet en waarom.

In Hoofdstuk~\ref{ch:conclusie}, tenslotte, wordt de conclusie gegeven en een antwoord geformuleerd op de onderzoeksvragen. Daarbij wordt ook een aanzet gegeven voor toekomstig onderzoek binnen dit domein.
\chapter{\IfLanguageName{dutch}{Stand van zaken}{State of the art}}%
\label{ch:stand-van-zaken}

% Tip: Begin elk hoofdstuk met een paragraaf inleiding die beschrijft hoe
% dit hoofdstuk past binnen het geheel van de bachelorproef. Geef in het
% bijzonder aan wat de link is met het vorige en volgende hoofdstuk.

% Pas na deze inleidende paragraaf komt de eerste sectiehoofding.

%Dit hoofdstuk bevat je literatuurstudie. De inhoud gaat verder op de inleiding, maar zal het onderwerp van de bachelorproef *diepgaand* uitspitten. De bedoeling is dat de lezer na lezing van dit hoofdstuk helemaal op de hoogte is van de huidige stand van zaken (state-of-the-art) in het %onderzoeksdomein. Iemand die niet vertrouwd is met het onderwerp, weet nu voldoende om de rest van het verhaal te kunnen volgen, zonder dat die er nog andere informatie moet over opzoeken \autocite{Pollefliet2011}.

%Je verwijst bij elke bewering die je doet, vakterm die je introduceert, enz.\ naar je bronnen. In \LaTeX{} kan dat met het commando \texttt{$\backslash${textcite\{\}}} of \texttt{$\backslash${autocite\{\}}}. Als argument van het commando geef je de ``sleutel'' van een ``record'' in een bibliografische databank in het Bib\LaTeX{}-formaat (een tekstbestand). Als je expliciet naar de auteur verwijst in de zin, gebruik je \texttt{$\backslash${}textcite\{\}}.
%Soms wil je de auteur niet expliciet vernoemen, dan gebruik je \texttt{$\backslash${}autocite\{\}}. In de volgende paragraaf een voorbeeld van elk.

%\textcite{Knuth1998} schreef een van de standaardwerken over sorteer- en zoekalgoritmen. Experten zijn het erover eens dat cloud computing een interessante opportuniteit vormen, zowel voor gebruikers als voor dienstverleners op vlak van informatietechnologie~\autocite{Creeger2009}.








\section{\IfLanguageName{dutch}{Dataportalen}{Research objective}}%
\label{sec:Dataportalen}

In deze literatuurstudie worden verschillende dataportals besproken die relevant zijn voor de maritieme industrie. Hierbij wordt gekeken naar hun gegevensbeheer, gegevensverwerking en gestandaardiseerde gegevensformaten. Een aantal van deze dataportals zijn gericht op specifieke regio's, terwijl anderen wereldwijd gegevens beschikbaar stellen.

EMODnet is zo'n dataportal dat zich richt op de Europese wateren en daarbij verschillende datasets aanbiedt die relevant zijn voor onderzoek en toepassingen in de maritieme sector.

Een ander dataportal dat besproken wordt is Marine Cadastre, dat zich richt op de Verenigde Staten en gegevens beschikbaar stelt met betrekking tot offshore-activiteiten en mariene hulpbronnen. Ook hier wordt gekeken naar het gegevensbeheer.
Daarnaast wordt ook de rol van de U.S. Army Corps of Engineers besproken in het kader van het Dredging Operations Technical Support Program en de beschikbare Dredging Data Extracts. 

Tot slot worden verschillende standaard gegevensformaten besproken, waaronder SeaDataNet, INSPIRE-richtlijn, GeoJSON, NetCDF, IHO S-102, Sensor Observation Service (SOS) en MCP.

\subsection{EMODnet}
EMODnet is een initiatief van de Europese Commissie. Het doel is om toegang te bieden tot maritieme gegevens van hoge kwaliteit. Het is ook gericht op het integreren en harmoniseren van deze gegevens. Dit is om beleidsmakers, wetenschappers, industrie en burgers te ondersteunen bij het nemen van beslissingen over maritieme zaken. EMODnet richt zich op een breed scala aan maritieme onderwerpen, waaronder mariene biodiversiteit, mariene geologie, fysische oceanografie en menselijke activiteiten.

Met betrekking tot menselijke activiteiten verzamelt en integreert EMODnet gegevens over onder andere scheepvaart, visserij, offshore-energieproductie en maritieme toerisme. Deze gegevens kunnen worden gebruikt voor een breed scala aan toepassingen, zoals maritieme ruimtelijke planning, monitoring van milieueffecten en beheer van maritieme hulpbronnen. Door de beschikbaarheid van deze gegevens te vergroten, draagt EMODnet bij aan een beter begrip en beheer van de impact van menselijke activiteiten op de zeeën en oceanen.

\subsubsection{Gegevensbeheer}
EMODnet hanteert een gestandaardiseerde aanpak voor gegevensbeheer. Deze aanpak omvat verschillende stappen om ervoor te zorgen dat de gegevens op een consistente en betrouwbare manier worden verzameld, verwerkt en beschikbaar worden gesteld.

De eerste stap is gegevensverzameling. EMODnet verzamelt gegevens uit verschillende bronnen, waaronder nationale en regionale autoriteiten, wetenschappelijke onderzoeksinstellingen en commerciële bedrijven. De gegevens worden verzameld met behulp van verschillende methoden, waaronder satellietbeelden, remote sensing en in situ waarnemingen.

Vervolgens worden de verzamelde gegevens onderworpen aan een reeks verwerkingsstappen om ervoor te zorgen dat ze nauwkeurig, consistent en betrouwbaar zijn. Dit kan onder andere het filteren van onjuiste gegevens, het omzetten van gegevens naar gestandaardiseerde formaten en het integreren van gegevens uit verschillende bronnen omvatten.

De gegevens worden opgeslagen in een gecentraliseerde database die door EMODnet wordt beheerd. Deze database is ontworpen om gemakkelijk toegankelijk te zijn voor gebruikers, met een reeks tools en interfaces om gegevensontdekking, visualisatie en analyse te vergemakkelijken.

Tot slot moedigt EMODnet het delen van gegevens en samenwerking aan tussen haar partners en gebruikers. Het netwerk biedt een scala aan tools en interfaces om gegevensuitwisseling te vergemakkelijken, waaronder gestandaardiseerde gegevensformaten, metadatastandaarden en gegevensontdekkingsportalen. Dankzij de focus van EMODnet op standaardisatie en kwaliteitscontrole zijn de gegevens nuttig voor een breed scala aan toepassingen, waaronder mariene ruimtelijke planning, milieueffectbeoordelingen en het beheer van mariene hulpbronnen.

\subsubsection{Gegevensverweking}

EMODnet gebruikt verschillende stappen om ervoor te zorgen dat de gegevens die het verzamelt en integreert, accuraat en betrouwbaar zijn en gemakkelijk toegankelijk zijn voor gebruikers. Dit omvat het proces van gegevensreiniging waarbij ongeldige of inconsistente gegevenspunten worden geïdentificeerd en verwijderd. Vervolgens worden de gegevens geformatteerd in een gestandaardiseerd formaat, zoals netCDF of CSV, om de consistentie en integratie van de gegevens met andere datasets te vergemakkelijken. EMODnet integreert gegevens van diverse bronnen. Deze bronnen omvatten nationale en regionale autoriteiten, wetenschappelijke onderzoeksinstellingen en commerciële bedrijven. EMODnet controleert deze gegevens op nauwkeurigheid en betrouwbaarheid. Dit doet het door middel van strenge kwaliteitscontrolemaatregelen. Bovendien wordt voor elke dataset metadata gemaakt die gebruikers helpt de gegevens te begrijpen en hun geschiktheid voor specifieke doeleinden te beoordelen. Tenslotte worden de verwerkte gegevens beschikbaar gesteld via verschillende interfaces en tools, zoals gegevensontdekkingsportals, webservices en API's.

\subsubsection{gestandaardiseerde gegevensformaten}
EMODnet maakt gebruik van verschillende gestandaardiseerde dataformaten, afhankelijk van het type data dat wordt verzameld. Dit omvat SeaDataNet, een formaat voor oceanografische gegevens met gestandaardiseerde variabelen, eenheden en metadatastandaarden. EMODnet houdt zich ook aan de INSPIRE-richtlijn van de Europese Unie voor gegevensinteroperabiliteit en metadata. Voor geografische gegevens gebruikt EMODnet GeoJSON, dat gegevens over bathymetrie, kustlijn en mariene habitats omvat. Daarnaast gebruikt EMODnet NetCDF, een veelgebruikt formaat voor het opslaan van wetenschappelijke gegevens in de oceanografische gemeenschap, voor roostergebaseerde datasets, zoals gegevens over de temperatuur van het zeewater en de stroming van de oceaan. Met deze gestandaardiseerde dataformaten kan EMODnet gegevens verzamelen, verwerken en delen op een nauwkeurige en consistente manier, wat van cruciaal belang is voor het bereiken van haar doelen van mariene ruimtelijke planning, milieueffectbeoordelingen en het beheer van mariene hulpbronnen.

\subsubsection{Data}

\subsection{}

\subsection{}

\section{Standaard gegevensformaten}
\label{sec:Standaard-gegevensformaten}

%%=============================================================================
%% Methodologie
%%=============================================================================

\chapter{\IfLanguageName{dutch}{Methodologie}{Methodology}}%
\label{ch:methodologie}

\textbf{Deel 1: Literatuurstudie over FAIR- en open-standaarden}

1.1 Definitie en belang van FAIR- en open-standaarden

Onderzoeken wat FAIR- en open-standaarden zijn en waarom ze belangrijk zijn in het kader van het onderzoek.
Literatuurstudie over FAIR-principes en open standaarden, zoals de Open Geospatial Consortium (OGC), ISO/IEC, en W3C-standaarden.

1.2 Ontstaan en evolutie van FAIR- en open-standaarden

Onderzoeken hoe FAIR- en open-standaarden zijn ontstaan en zich hebben ontwikkeld tot wat ze vandaag zijn.
Literatuurstudie over de geschiedenis van FAIR- en open-standaarden, zoals de ontwikkeling van de OGC-standaarden en de ISO/IEC-normen.

1.3 Analyse van bestandsformaten

Op basis van de literatuurstudie, bepalen welke bestandsformaten in aanmerking komen voor het eindresultaat.
Beoordelen welke bestandsformaten voldoen aan de FAIR-principes en open-standaarden.

\textbf{Deel 2: Interviews met betrokken personen}

2.1 Interviews met verstrekkers van baggergegevens

Interviews afnemen met de huidige verstrekkers van baggergegevens om te achterhalen welke bestandsformaten zij momenteel gebruiken en waarom.
Vragen stellen over eventuele problemen met de huidige bestandsformaten.

2.2 Interviews met opdrachtgevers en wetenschappelijke experts

Interviews afnemen met de betrokken opdrachtgevers en wetenschappelijke experts om te achterhalen welke bestandsformaten zij voorkeur voor hebben en waarom.
Vergelijken van verschillende dataformaten.

\textbf{Deel 3: Implementatie van de data in het datasysteem of proof-of-concept}

3.1 Implementatie in datasysteem

Op basis van de resultaten uit de literatuurstudie en de interviews, implementeren van de uniforme data in het datasysteem.
Overleggen met de verantwoordelijke van het datasysteem om te bepalen of de implementatie effectief is en er geen problemen ontstaan.

3.2 Proof-of-concept

Indien de implementatie in het datasysteem niet mogelijk is, zal een proof-of-concept worden ontwikkeld.
Evalueren of de uniforme data effectief kan worden opgenomen in het proof-of-concept.

%% TODO: Hoe ben je te werk gegaan? Verdeel je onderzoek in grote fasen, en
%% licht in elke fase toe welke stappen je gevolgd hebt. Verantwoord waarom je
%% op deze manier te werk gegaan bent. Je moet kunnen aantonen dat je de best
%% mogelijke manier toegepast hebt om een antwoord te vinden op de
%% onderzoeksvraag.

\lipsum[21-25]



% Voeg hier je eigen hoofdstukken toe die de ``corpus'' van je bachelorproef
% vormen. De structuur en titels hangen af van je eigen onderzoek. Je kan bv.
% elke fase in je onderzoek in een apart hoofdstuk bespreken.

\chapter{\IfLanguageName{dutch}Data}{State of the art}}%
\label{ch:data}


\chapter{\IfLanguageName{dutch}{Standaard formaat}{State of the art}}%
\label{ch:standaard-formaat}


%...

\input{conclusie}

%---------- Bijlagen -----------------------------------------------------------

\appendix

\chapter{Onderzoeksvoorstel}

Het onderwerp van deze bachelorproef is gebaseerd op een onderzoeksvoorstel dat vooraf werd beoordeeld door de promotor. Dat voorstel is opgenomen in deze bijlage.

%% TODO: 
%\section*{Samenvatting}

% Kopieer en plak hier de samenvatting (abstract) van je onderzoeksvoorstel.

% Verwijzing naar het bestand met de inhoud van het onderzoeksvoorstel
%---------- Inleiding ---------------------------------------------------------

\section{Introductie}%
\label{sec:introductie}

Het dataportaal ScheldeMonitor \footnote{https://www.scheldemonitor.org/nl} probeert alle wetenschappelijke data en informatie van het 
Schelde- estuarium op één plaats samen te brengen. Deze data wordt verzameld door verschillende monitoringprogramma's van verscheidene bronnen. De data bestaat uit variërende gegevens over troebelheid, nutriënten, verontreinigende stoffen en verschillende fysische parameters. Al deze parameters worden echter sterk beïnvloed door menselijke activiteiten zoals baggeren en scheepvaart. Het zou dus een grote toegevoegde waarde zijn om ook de data over deze menselijke activiteiten in het portaal op te nemen.

Scheepvaartdata is al gestandaardiseerd en kan vrij goed worden vastgelegd via een AIS-platform. Baggergegevens zijn echter nog nooit gestroomlijnd geweest voor de Schelde. Informatie over de locatie, het soort activiteit, de hoeveelheid gebaggerd materiaal en andere data worden gemeten en vastgelegd door verschillende aanbieders. Dit zijn vaak particuliere aannemers. Het resulteert hiervan is dat er een overvloed van verschillende formaten en databestanden aanwezig zijn. Deze worden voor het overgrote deel opgeslagen in interne systemen van het verantwoordelijke bestuursinstituut. Toegang tot of verstrekking van deze bestanden is alleen mogelijk via handmatige aanvragen.

Welk uniform bestandsformaat zal het best gebruikt worden om deze data zo vlot mogelijk te implementeren binnen het dataportaal ScheldeMonitor? Tijdens deze thesis zal hiervoor een antwoord gezocht worden zodat VLIZ (Vlaams Instituut voor de Zee) hun portaal kan uitbreiden. Als resultaat zal er dus een uniform dataformaat zijn die voldoet aan FAIR- en open-standaarden en die efficiënt data kan importeren in het datasysteem.

%---------- Stand van zaken ---------------------------------------------------

\section{Literatuurstudie}%
\label{sec:state-of-the-art}

\subsection{VLIZ}
De creatie van het Vlaams Instituut voor de Zee (VLIZ) had als doel een duidelijk identificeerbaar aanspreekpunt te maken in Vlaanderen \autocite{Mees2012}. Dit zodat de zichtbaarheid van het zeewetenschappelijk onderzoek kon vergroot worden op regionaal en internationaal niveau. Het effectief ontstaan van de organisatie gebeurde op 1 oktober 1999. Dit werd verwezenlijkt door de steun van de Vlaamse Regering, de provincie West-Vlaanderen en het Fonds voor Wetenschappelijk Onderzoek Vlaanderen. Doorheen de jaren groeide VLIZ in haar rol als ondersteunende instantie van het marien wetenschappelijk onderzoek in Vlaanderen. Het doet dit door een variatie van diensten aan te bieden. Wetenschappers hebben zelf soms minder tijd of middelen voor bepaalde onderzoeken uit te voeren die toch belangrijk kunnen zijn. VLIZ kan hier dan het onderzoek op haar nemen. Tegenwoordig zijn de belangrijkste taken van VLIZ de volgende: het beheer van marien wetenschappelijke data en informatie, wetenschapscommunicatie, het vertalen van onderzoeksresultaten naar de overheid en het grote publiek toe, het ondersteunen van wetenschappelijke publicaties, de integratie in internationale netwerken en het organiseren van workshops en symposia.

\subsection{de Schelde}
Een van de actieve onderzoeken van VLIZ is Het Schelde-estuarium. De Schelde heeft heel wat belangrijke functies: water- en sedimentafvoer, scheepvaart, visserij, recreatie en natuurfunctie \autocite{Melrel1992}. Estuaria zijn zeer dynamische en kunnen snel veranderen. Estuaria zijn ook zeer productief en ondersteunen onder andere ook ecosysteemfuncties \autocite{Meire2005}: bio- en geo- chemische kringloop en verplaatsing van nutriënten, beperking van overstromingen, behoud van biodiversiteit en biologische productie. De menselijke invloed op estuaria is daarom ook zeer belangrijk om te controleren. Het Schelde-estuarium, gelegen in het noordwesten van Vlaanderen (België) en het zuidwesten van Nederland, is het benedenstroomse deel van de Schelde. Het totale bekkengebied bedraagt 21 863 vierkante kilometer \autocite{Peeters2015}. Deze oppervlakte is verdeeld over Frankrijk, België en Nederland. Van de bron (Noord-Frankrijk) tot in de monding (Noordzee) heeft de Schelde een lengte van 355 kilometer. Vanaf de grens tussen Vlaanderen en Nederland verbreedt de rivier en ontstaat het brakke estuarium. Dit wordt ook wel de Westerschelde genoemd.

\subsection{data}
Er zijn heel wat invloeden die een effect kunnen hebben op de Schelde. VLIZ heeft dan ook heel verschillende data uit verschillende bronnen. Het is belangrijk om deze data zo algemeen mogelijk op te stellen en te behouden. Dit zodat er logische conclusies kunnen getrokken worden uit de verschillende data. Als de data niet algemeen is opgesteld, wordt het al een heel stuk moeilijker om data te vergelijken of grafieken op te maken van soortgelijke data. Best practices kunnen hierbij helpen. Er zal gestreefd worden naar FAIR-standaarden (findability, accessibility, interoperability, and reusability). Databronnen worden tegenwoordig met veel verschillende datamodellen opgesteld \autocite{Stupnikov2018}. Enkele voorbeelden van verschillende modellen zijn: het traditioneel relationeel model, het semantische model zoals RDF en OWL en modellen voor
semi-gestructureerde gegevens zoals NoSQL, XML en JSON. Deze formaten kunnen ook zeer verschillende datamanipulatie opleveren. Het is dus belangrijk om een eenduidig model te gebruiken tijdens het verwerken van data.

% Voor literatuurverwijzingen zijn er twee belangrijke commando's:
% \autocite{KEY} => (Auteur, jaartal) Gebruik dit als de naam van de auteur
%   geen onderdeel is van de zin.
% \textcite{KEY} => Auteur (jaartal)  Gebruik dit als de auteursnaam wel een
%   functie heeft in de zin (bv. ``Uit onderzoek door Doll & Hill (1954) bleek
%   ...'')

%---------- Methodologie ------------------------------------------------------
\section{Methodologie}%
\label{sec:methodologie}

Deel 1 van het onderzoek zal bestaan uit een literatuurstudie over FAIR- en open-standaarden. In dit deel zal er onderzocht worden wat FAIR- en open-standaarden zijn en waarom ze belangrijk zijn. Het ontstaan en de evolutie van deze standaarden zal hier ook besproken worden. Dit zal de basis vormen voor welke bestandsformaten een kandidaat kunnen zijn voor het eindresultaat.

Deel 2 van het onderzoek zal bestaan uit interviews met betrokken personen. Er zullen interviews zijn met de huidige verstrekkers van baggergegevens, de betrokken opdrachtgevers en wetenschappelijke experts. Hier zal duidelijk worden wat de redenen zijn dat deze personen bepaalde bestandsformaten gebruiken.

Deel 3 van het onderzoek zal bestaan uit het implementeren van de data in het datasysteem of een proof-of-concept. In dit deel van het onderzoek zal gekeken worden of de uniforme data effectief implementeerbaar is in het datasysteem. Hierover zal nog uitleg gegeven worden door de verantwoordelijke van het datasysteem.

%---------- Verwachte resultaten ----------------------------------------------
\section{Verwacht resultaat, conclusie}%
\label{sec:verwachte_resultaten}

Als eindresultaat zal er dus een uniform dataformaat zijn die voldoet aan FAIR- en open-standaarden en die efficiënt data kan importeren in het datasysteem. Dit dataformaat zal opgedrongen worden aan de verschillende leveranciers en contractanten van VLIZ. Er zal ook gekeken worden of er al een proof-of-concept mogelijk is op basis van dit resultaat.

Op basis van deze resultaten zal het mogelijk zijn om data van de menselijke activiteiten in de Schelde op te nemen in het dataportaal ScheldeMonitor.



%%---------- Andere bijlagen --------------------------------------------------
% TODO: Voeg hier eventuele andere bijlagen toe. Bv. als je deze BP voor de
% tweede keer indient, een overzicht van de verbeteringen t.o.v. het origineel.
%\input{...}

%%---------- Backmatter, referentielijst ---------------------------------------

\backmatter{}

\setlength\bibitemsep{2pt} %% Add Some space between the bibliograpy entries
\printbibliography[heading=bibintoc]

\end{document}
