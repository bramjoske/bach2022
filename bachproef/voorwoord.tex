%%=============================================================================
%% Voorwoord
%%=============================================================================

\chapter*{\IfLanguageName{dutch}{Woord vooraf}{Preface}}%
\label{ch:voorwoord}

%% TODO:
%% Het voorwoord is het enige deel van de bachelorproef waar je vanuit je
%% eigen standpunt (``ik-vorm'') mag schrijven. Je kan hier bv. motiveren
%% waarom jij het onderwerp wil bespreken.
%% Vergeet ook niet te bedanken wie je geholpen/gesteund/... heeft

Beste lezers,

Met trots presenteer ik mijn bachelorscriptie over uniforme dataformaten voor gegevens over menselijke activiteiten in de Schelde. Als student aan de Hogeschool Gent heb ik de afgelopen maanden intensief samengewerkt met mijn promotor, Chantal Teerlinck, en mijn co-promotor, Charlotte Dhondt van het Vlaams Instituut voor de Zee (VLIZ). Zij hebben mij gedurende het hele proces begeleid en ondersteund in mijn onderzoek.

De Schelde is een belangrijke waterweg die door meerdere landen stroomt en een cruciale rol speelt in de economie, natuur en leefbaarheid van de omgeving. Het verzamelen en delen van gegevens over menselijke activiteiten in de Schelde is van groot belang voor het monitoren en beheren van deze waterweg. Echter, het gebruik van verschillende dataformaten door verschillende instanties maakt het lastig om deze gegevens goed te vergelijken en te analyseren.

In mijn scriptie heb ik me gericht op het ontwikkelen van uniforme dataformaten voor gegevens over menselijke activiteiten in de Schelde. Ik heb onderzocht welke factoren van invloed zijn op de uniformiteit van deze dataformaten en hoe deze factoren kunnen worden aangepakt. Hierbij heb ik gebruik gemaakt van verschillende bronnen, waaronder literatuuronderzoek en interviews met experts.

Het resultaat van mijn onderzoek is een set van aanbevelingen voor uniforme dataformaten voor gegevens over menselijke activiteiten in de Schelde. Deze aanbevelingen kunnen worden gebruikt door instanties die zich bezighouden met het verzamelen en delen van deze gegevens, om zo de uniformiteit en vergelijkbaarheid van de data te verbeteren.

Ik hoop dat mijn scriptie bijdraagt aan een beter begrip van het belang van uniforme dataformaten voor gegevens over menselijke activiteiten in de Schelde, en dat de aanbevelingen die ik heb geformuleerd worden gebruikt om de samenwerking tussen verschillende instanties te verbeteren.

Met vriendelijke groet,

Bram Verbanck