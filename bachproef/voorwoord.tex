%%=============================================================================
%% Voorwoord
%%=============================================================================

\chapter*{\IfLanguageName{dutch}{Woord vooraf}{Preface}}%
\label{ch:voorwoord}

%% TODO:
%% Het voorwoord is het enige deel van de bachelorproef waar je vanuit je
%% eigen standpunt (``ik-vorm'') mag schrijven. Je kan hier bv. motiveren
%% waarom jij het onderwerp wil bespreken.
%% Vergeet ook niet te bedanken wie je geholpen/gesteund/... heeft

Beste lezers,

Met grote trots presenteer ik mijn thesis over gestandaardiseerde gegevensindelingen voor informatie over menselijke activiteiten in de Schelde. Als student aan de Hogeschool Gent heb ik intensief samengewerkt met mijn promotor, Chantal Teerlinck en mijn co-promotor, Charlotte Dhondt van het Vlaams Instituut voor de Zee (VLIZ). Zij hebben mij gedurende het hele proces begeleid en ondersteund bij mijn onderzoek.

Daarnaast wil ik graag mijn oprechte dank uitspreken aan Paul Focke en Jelle Rondelez van VLIZ, die ook een rol hebben gespeeld in de begeleiding van mijn thesis. Hun expertise en betrokkenheid hebben mijn onderzoek verrijkt.

Ik wil tevens mijn waardering uitspreken voor Patricia Cabrera van VLIZ, die haar tijd heeft genomen om bepaalde zaken door te sturen. Haar bijdrage heeft mijn onderzoek verder versterkt.

Ook Suffis Jürgen van het Departement Openbare Werken (MOW) verdient mijn dank voor zijn waardevolle tijd en inspanningen bij het doorsturen van enkele bestanden die essentieel waren voor mijn onderzoek.

De Schelde is een belangrijke waterweg die door meerdere landen stroomt en een cruciale rol speelt in de economie, natuur en leefbaarheid van de omgeving. Het verzamelen en delen van gegevens over menselijke activiteiten in de Schelde is van groot belang voor het monitoren en beheren van deze waterweg. Helaas bemoeilijkt het gebruik van verschillende gegevensindelingen door verschillende organisaties de vergelijking en analyse van deze gegevens.

In mijn thesis heb ik mij gericht op het ontwikkelen van gestandaardiseerde gegevensindelingen voor informatie over menselijke activiteiten in de Schelde. Ik heb onderzocht hoe gegevens momenteel worden opgemaakt voor VLIZ en tevens gekeken naar de manier waarop soortgelijke gegevens buiten VLIZ worden weergegeven. Hiervoor heb ik gebruikgemaakt van diverse bronnen, waaronder literatuuronderzoek en overleg met deskundigen.

Het resultaat van mijn onderzoek is een reeks aanbevelingen voor gestandaardiseerde gegevensindelingen voor informatie over menselijke activiteiten in de Schelde. Deze aanbevelingen zijn specifiek opgemaakt voor VLIZ, maar kunnen ook worden gebruikt door organisaties die zich bezighouden met het verzamelen en delen van deze gegevens. De doelstelling is om zo de uniformiteit en vergelijkbaarheid van de gegevens te verbeteren.

Ik hoop dat mijn scriptie bijdraagt aan een beter begrip van het belang van gestandaardiseerde gegevensindelingen voor informatie over menselijke activiteiten in de Schelde. Tevens hoop ik dat de geformuleerde aanbevelingen worden gebruikt om de samenwerking tussen verschillende organisaties te bevorderen.

Met vriendelijke groet,

Bram Verbanck