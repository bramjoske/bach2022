%%=============================================================================
%% Samenvatting
%%=============================================================================

% TODO: De "abstract" of samenvatting is een kernachtige (~ 1 blz. voor een
% thesis) synthese van het document.
%
% Een goede abstract biedt een kernachtig antwoord op volgende vragen:
%
% 1. Waarover gaat de bachelorproef?
% 2. Waarom heb je er over geschreven?
% 3. Hoe heb je het onderzoek uitgevoerd?
% 4. Wat waren de resultaten? Wat blijkt uit je onderzoek?
% 5. Wat betekenen je resultaten? Wat is de relevantie voor het werkveld?
%
% Daarom bestaat een abstract uit volgende componenten:
%
% - inleiding + kaderen thema
% - probleemstelling
% - (centrale) onderzoeksvraag
% - onderzoeksdoelstelling
% - methodologie
% - resultaten (beperk tot de belangrijkste, relevant voor de onderzoeksvraag)
% - conclusies, aanbevelingen, beperkingen
%
% LET OP! Een samenvatting is GEEN voorwoord!

%%---------- Nederlandse samenvatting -----------------------------------------
%
% TODO: Als je je bachelorproef in het Engels schrijft, moet je eerst een
% Nederlandse samenvatting invoegen. Haal daarvoor onderstaande code uit
% commentaar.
% Wie zijn bachelorproef in het Nederlands schrijft, kan dit negeren, de inhoud
% wordt niet in het document ingevoegd.

\IfLanguageName{english}{%
\selectlanguage{dutch}
\chapter*{Samenvatting}
\lipsum[1-4]
\selectlanguage{english}
}{}

%%---------- Samenvatting -----------------------------------------------------
% De samenvatting in de hoofdtaal van het document

\chapter*{\IfLanguageName{dutch}{Samenvatting}{Abstract}}

Deze bachelorscriptie gaat over uniforme dataformaten voor gegevens over menselijke activiteiten in de Schelde. Het doel van dit onderzoek is om aanbevelingen te formuleren voor het verbeteren van de uniformiteit van deze dataformaten. Het probleem is dat verschillende instanties verschillende dataformaten gebruiken, waardoor het lastig is om deze gegevens goed te vergelijken en te analyseren. De centrale onderzoeksvraag luidt dan ook: 'Hoe kunnen uniforme dataformaten worden ontwikkeld voor gegevens over menselijke activiteiten in de Schelde?'

Om deze vraag te beantwoorden is een literatuuronderzoek uitgevoerd en zijn interviews gehouden met experts. Hieruit zijn factoren naar voren gekomen die van invloed zijn op de uniformiteit van de dataformaten. Op basis hiervan zijn aanbevelingen geformuleerd voor het ontwikkelen van uniforme dataformaten. De belangrijkste resultaten zijn dat het van belang is om een gemeenschappelijk referentiekader te hanteren, afspraken te maken over de definities van gegevens en metadata, en te zorgen voor een gestandaardiseerde structuur van de data.

De relevantie van deze resultaten voor het werkveld is dat uniforme dataformaten de samenwerking tussen verschillende instanties kunnen verbeteren en het verzamelen en delen van gegevens over menselijke activiteiten in de Schelde kunnen faciliteren. Dit kan bijdragen aan het monitoren en beheren van de Schelde als belangrijke waterweg. De conclusie is dat er behoefte is aan uniforme dataformaten en dat deze aanbevelingen kunnen helpen om deze uniformiteit te bereiken.
