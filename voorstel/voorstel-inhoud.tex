%---------- Inleiding ---------------------------------------------------------

\section{Introductie}%
\label{sec:introductie}

Het dataportaal ScheldeMonitor \footnote{https://www.scheldemonitor.org/nl} probeert alle wetenschappelijke data en informatie van het 
Schelde- estuarium op één plaats samen te brengen. Deze data wordt verzameld door verschillende monitoringprogramma's van verscheidene bronnen. De data bestaat uit variërende gegevens over troebelheid, nutriënten, verontreinigende stoffen en verschillende fysische parameters. Al deze parameters worden echter sterk beïnvloed door menselijke activiteiten zoals baggeren en scheepvaart. Het zou dus een grote toegevoegde waarde zijn om ook de data over deze menselijke activiteiten in het portaal op te nemen.

Scheepvaartdata is al gestandaardiseerd en kan vrij goed worden vastgelegd via een AIS-platform. Baggergegevens zijn echter nog nooit gestroomlijnd geweest voor de Schelde. Informatie over de locatie, het soort activiteit, de hoeveelheid gebaggerd materiaal en andere data worden gemeten en vastgelegd door verschillende aanbieders. Dit zijn vaak particuliere aannemers. Het resulteert hiervan is dat er een overvloed van verschillende formaten en databestanden aanwezig zijn. Deze worden voor het overgrote deel opgeslagen in interne systemen van het verantwoordelijke bestuursinstituut. Toegang tot of verstrekking van deze bestanden is alleen mogelijk via handmatige aanvragen.

Welk uniform bestandsformaat zal het best gebruikt worden om deze data zo vlot mogelijk te implementeren binnen het dataportaal ScheldeMonitor? Tijdens deze thesis zal hiervoor een antwoord gezocht worden zodat VLIZ (Vlaams Instituut voor de Zee) hun portaal kan uitbreiden. Als resultaat zal er dus een uniform dataformaat zijn die voldoet aan FAIR- en open-standaarden en die efficiënt data kan importeren in het datasysteem.

%---------- Stand van zaken ---------------------------------------------------

\section{Literatuurstudie}%
\label{sec:state-of-the-art}

\subsection{VLIZ}
De creatie van het Vlaams Instituut voor de Zee (VLIZ) had als doel een duidelijk identificeerbaar aanspreekpunt te maken in Vlaanderen \autocite{Mees2012}. Dit zodat de zichtbaarheid van het zeewetenschappelijk onderzoek kon vergroot worden op regionaal en internationaal niveau. Het effectief ontstaan van de organisatie gebeurde op 1 oktober 1999. Dit werd verwezenlijkt door de steun van de Vlaamse Regering, de provincie West-Vlaanderen en het Fonds voor Wetenschappelijk Onderzoek Vlaanderen. Doorheen de jaren groeide VLIZ in haar rol als ondersteunende instantie van het marien wetenschappelijk onderzoek in Vlaanderen. Het doet dit door een variatie van diensten aan te bieden. Wetenschappers hebben zelf soms minder tijd of middelen voor bepaalde onderzoeken uit te voeren die toch belangrijk kunnen zijn. VLIZ kan hier dan het onderzoek op haar nemen. Tegenwoordig zijn de belangrijkste taken van VLIZ de volgende: het beheer van marien wetenschappelijke data en informatie, wetenschapscommunicatie, het vertalen van onderzoeksresultaten naar de overheid en het grote publiek toe, het ondersteunen van wetenschappelijke publicaties, de integratie in internationale netwerken en het organiseren van workshops en symposia.

\subsection{de Schelde}
Een van de actieve onderzoeken van VLIZ is Het Schelde-estuarium. De Schelde heeft heel wat belangrijke functies: water- en sedimentafvoer, scheepvaart, visserij, recreatie en natuurfunctie \autocite{Melrel1992}. Estuaria zijn zeer dynamische en kunnen snel veranderen. Estuaria zijn ook zeer productief en ondersteunen onder andere ook ecosysteemfuncties \autocite{Meire2005}: bio- en geo- chemische kringloop en verplaatsing van nutriënten, beperking van overstromingen, behoud van biodiversiteit en biologische productie. De menselijke invloed op estuaria is daarom ook zeer belangrijk om te controleren. Het Schelde-estuarium, gelegen in het noordwesten van Vlaanderen (België) en het zuidwesten van Nederland, is het benedenstroomse deel van de Schelde. Het totale bekkengebied bedraagt 21 863 vierkante kilometer \autocite{Peeters2015}. Deze oppervlakte is verdeeld over Frankrijk, België en Nederland. Van de bron (Noord-Frankrijk) tot in de monding (Noordzee) heeft de Schelde een lengte van 355 kilometer. Vanaf de grens tussen Vlaanderen en Nederland verbreedt de rivier en ontstaat het brakke estuarium. Dit wordt ook wel de Westerschelde genoemd.

\subsection{data}
Er zijn heel wat invloeden die een effect kunnen hebben op de Schelde. VLIZ heeft dan ook heel verschillende data uit verschillende bronnen. Het is belangrijk om deze data zo algemeen mogelijk op te stellen en te behouden. Dit zodat er logische conclusies kunnen getrokken worden uit de verschillende data. Als de data niet algemeen is opgesteld, wordt het al een heel stuk moeilijker om data te vergelijken of grafieken op te maken van soortgelijke data. Best practices kunnen hierbij helpen. Er zal gestreefd worden naar FAIR-standaarden (findability, accessibility, interoperability, and reusability). Databronnen worden tegenwoordig met veel verschillende datamodellen opgesteld \autocite{Stupnikov2018}. Enkele voorbeelden van verschillende modellen zijn: het traditioneel relationeel model, het semantische model zoals RDF en OWL en modellen voor
semi-gestructureerde gegevens zoals NoSQL, XML en JSON. Deze formaten kunnen ook zeer verschillende datamanipulatie opleveren. Het is dus belangrijk om een eenduidig model te gebruiken tijdens het verwerken van data.

% Voor literatuurverwijzingen zijn er twee belangrijke commando's:
% \autocite{KEY} => (Auteur, jaartal) Gebruik dit als de naam van de auteur
%   geen onderdeel is van de zin.
% \textcite{KEY} => Auteur (jaartal)  Gebruik dit als de auteursnaam wel een
%   functie heeft in de zin (bv. ``Uit onderzoek door Doll & Hill (1954) bleek
%   ...'')

%---------- Methodologie ------------------------------------------------------
\section{Methodologie}%
\label{sec:methodologie}

Deel 1 van het onderzoek zal bestaan uit een literatuurstudie over FAIR- en open-standaarden. In dit deel zal er onderzocht worden wat FAIR- en open-standaarden zijn en waarom ze belangrijk zijn. Het ontstaan en de evolutie van deze standaarden zal hier ook besproken worden. Dit zal de basis vormen voor welke bestandsformaten een kandidaat kunnen zijn voor het eindresultaat.

Deel 2 van het onderzoek zal bestaan uit interviews met betrokken personen. Er zullen interviews zijn met de huidige verstrekkers van baggergegevens, de betrokken opdrachtgevers en wetenschappelijke experts. Hier zal duidelijk worden wat de redenen zijn dat deze personen bepaalde bestandsformaten gebruiken.

Deel 3 van het onderzoek zal bestaan uit het implementeren van de data in het datasysteem of een proof-of-concept. In dit deel van het onderzoek zal gekeken worden of de uniforme data effectief implementeerbaar is in het datasysteem. Hierover zal nog uitleg gegeven worden door de verantwoordelijke van het datasysteem.

%---------- Verwachte resultaten ----------------------------------------------
\section{Verwacht resultaat, conclusie}%
\label{sec:verwachte_resultaten}

Als eindresultaat zal er dus een uniform dataformaat zijn die voldoet aan FAIR- en open-standaarden en die efficiënt data kan importeren in het datasysteem. Dit dataformaat zal opgedrongen worden aan de verschillende leveranciers en contractanten van VLIZ. Er zal ook gekeken worden of er al een proof-of-concept mogelijk is op basis van dit resultaat.

Op basis van deze resultaten zal het mogelijk zijn om data van de menselijke activiteiten in de Schelde op te nemen in het dataportaal ScheldeMonitor.

