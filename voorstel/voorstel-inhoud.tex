%---------- Inleiding ---------------------------------------------------------

\section{Introductie}%
\label{sec:introductie}

Het dataportaal ScheldeMonitor \footnote{https://www.scheldemonitor.org/nl} probeert alle wetenschappelijke data en informatie van het 
Schelde- estuarium op één plaats samen te brengen. Deze data wordt verzameld door verschillende monitoringprogramma's van verscheidene bronnen. De data bestaat uit variërende gegevens over troebelheid, nutriënten, verontreinigende stoffen en verschillende fysische parameters. Al deze parameters worden echter sterk beïnvloed door menselijke activiteiten zoals baggeren en scheepvaart. Het zou dus een grote toegevoegde waarde zijn om ook de data over deze menselijke activiteiten in het portaal op te nemen.

Scheepvaartdata is al gestandaardiseerd en kan vrij goed worden vastgelegd via een AIS-platform. Baggergegevens zijn echter nog nooit gestroomlijnd geweest voor de Schelde. Informatie over de locatie, het soort activiteit, de hoeveelheid gebaggerd materiaal en andere data worden gemeten en vastgelegd door verschillende aanbieders. Dit zijn vaak particuliere aannemers. Het resulteert hiervan is dat er een overvloed van verschillende formaten en databestanden aanwezig zijn. Deze worden voor het overgrote deel opgeslagen in interne systemen van het verantwoordelijke bestuursinstituut. Toegang tot of verstrekking van deze bestanden is alleen mogelijk via handmatige aanvragen.

Welk uniform bestandsformaat zal het best gebruikt worden om deze data zo vlot mogelijk te implementeren binnen het dataportaal ScheldeMonitor? Tijdens deze thesis zal hiervoor een antwoord gezocht worden zodat VLIZ (Vlaams Instituut voor de Zee) hun portaal kan uitbreiden. Als resultaat zal er dus een uniform dataformaat zijn die voldoet aan FAIR- en open-standaarden en die efficiënt data kan importeren in het datasysteem.

%---------- Stand van zaken ---------------------------------------------------

\section{Literatuurstudie}%
\label{sec:state-of-the-art}

\subsection{VLIZ}
De creatie van het Vlaams Instituut voor de Zee (VLIZ) had als doel een duidelijk identificeerbaar aanspreekpunt te maken in Vlaanderen \autocite{Mees2012}. Dit zodat de zichtbaarheid van het zeewetenschappelijk onderzoek kon vergroot worden op regionaal en internationaal niveau. Het effectief ontstaan van de organisatie gebeurde op 1 oktober 1999. Dit werd verwezenlijkt door de steun van de Vlaamse Regering, de provincie West-Vlaanderen en het Fonds voor Wetenschappelijk Onderzoek Vlaanderen. Doorheen de jaren groeide VLIZ in haar rol als ondersteunende instantie van het marien wetenschappelijk onderzoek in Vlaanderen. Het doet dit door een variatie van diensten aan te bieden. Wetenschappers hebben zelf soms minder tijd of middelen voor bepaalde onderzoeken uit te voeren die toch belangrijk kunnen zijn. VLIZ kan hier dan het onderzoek op haar nemen. Tegenwoordig zijn de belangrijkste taken van VLIZ de volgende: het beheer van marien wetenschappelijke data en informatie, wetenschapscommunicatie, het vertalen van onderzoeksresultaten naar de overheid en het grote publiek toe, het ondersteunen van wetenschappelijke publicaties, de integratie in internationale netwerken en het organiseren van workshops en symposia.

\subsection{de Schelde}
Een van de actieve onderzoeken van VLIZ is Het Schelde-estuarium. De Schelde heeft heel wat belangrijke functies: water- en sedimentafvoer, scheepvaart, visserij, recreatie en natuurfunctie \autocite{Melrel1992}. Estuaria zijn zeer dynamische en kunnen snel veranderen. Estuaria zijn ook zeer productief en ondersteunen onder andere ook ecosysteemfuncties \autocite{Meire2005}: bio- en geo- chemische kringloop en verplaatsing van nutriënten, beperking van overstromingen, behoud van biodiversiteit en biologische productie. De menselijke invloed op estuaria is daarom ook zeer belangrijk om te controleren. Het Schelde-estuarium, gelegen in het noordwesten van Vlaanderen (België) en het zuidwesten van Nederland, is het benedenstroomse deel van de Schelde. Het totale bekkengebied bedraagt 21 863 vierkante kilometer \autocite{Peeters2015}. Deze oppervlakte is verdeeld over Frankrijk, België en Nederland. Van de bron (Noord-Frankrijk) tot in de monding (Noordzee) heeft de Schelde een lengte van 355 kilometer. Vanaf de grens tussen Vlaanderen en Nederland verbreedt de rivier en ontstaat het brakke estuarium. Dit wordt ook wel de Westerschelde genoemd.

\subsection{data}
Er zijn heel wat invloeden die een effect kunnen hebben op de Schelde. VLIZ heeft dan ook heel wat verschillende data uit verschillende bronnen. Het is belangrijk om deze data zo algemeen mogelijk op te stellen en te behouden. Dit zodat er logische conclusies kunnen getrokken worden uit de verschillende data. Als de data niet algemeen is opgesteld, wordt het al een heel stuk moeilijker om data te vergelijken of grafieken op te maken van soortgelijke data. Best practices kunnen hierbij helpen. Er zal gestreefd worden naar FAIR-standaarden (findability, accessibility, interoperability, and reusability). 

Met verschillende gegevensformaten in omloop, kan het kiezen van het juiste gegevensformaat voor opslag echter een uitdaging zijn. Hieronder staan enkele verschillende gegevensformaten die in een database kunnen worden opgeslagen.

\begin{enumerate}
    \item Gestructureerde gegevens: Deze gegevens zijn zeer georganiseerd en bestaan in een vast formaat \autocite{Dantoni2022}. Het omvat gegevenstypen zoals datums, nummers en tekst. Gestructureerde gegevens zijn gemakkelijk te doorzoeken, filteren en analyseren, waardoor het een populaire keuze is voor database-opslag.
    
    \item Ongestructureerde gegevens: Deze gegevens bestaan in een niet-standaard formaat, waardoor het moeilijk te organiseren en doorzoeken is \autocite{IBMCloudEducation2021}. Voorbeelden hiervan zijn video's, afbeeldingen en audiobestanden. Het opslaan van ongestructureerde gegevens kan een uitdaging zijn, maar kan waardevolle inzichten opleveren bij analyse.
    
    \item Semi-gestructureerde gegevens: Deze gegevens zijn gedeeltelijk georganiseerd en kunnen in verschillende formaten worden opgeslagen \autocite{ihritik2021}. Voorbeelden hiervan zijn XML-, JSON- en CSV-bestanden. Semi-gestructureerde gegevens kunnen lastig zijn om te beheren, maar kunnen zeer nuttig zijn bij analyse.
    
    \item Binaire gegevens: Deze gegevens bestaan uit machinaal leesbare code en omvatten bestanden zoals PDF's en afbeeldingen \autocite{Heddings2018}. Het opslaan van binaire gegevens kan een uitdaging zijn, maar kan zeer waardevol zijn bij analyse.
    
    \item Metadata: Deze gegevens bieden informatie over andere gegevens, zoals gegevenstype, auteur en aanmaakdatum  \autocite{TECHOPEDIA2017}. Het is essentieel voor het beheren en analyseren van gegevens.
\end{enumerate}

% Voor literatuurverwijzingen zijn er twee belangrijke commando's:
% \autocite{KEY} => (Auteur, jaartal) Gebruik dit als de naam van de auteur
%   geen onderdeel is van de zin.
% \textcite{KEY} => Auteur (jaartal)  Gebruik dit als de auteursnaam wel een
%   functie heeft in de zin (bv. ``Uit onderzoek door Doll & Hill (1954) bleek
%   ...'')

%---------- Methodologie ------------------------------------------------------
\section{Methodologie}%
\label{sec:methodologie}

Het onderzoek zal uit drie delen bestaan. In het eerste deel zal een literatuurstudie worden uitgevoerd over FAIR- en open-standaarden, waarbij de nadruk ligt op de verkenning van deze standaarden en de redenen voor hun belang. Ook de ontwikkeling en evolutie van deze standaarden zal worden besproken, als basis voor het bepalen van de bestandsformaten die kunnen worden gebruikt in de uiteindelijke resultaten.

Het tweede deel van het onderzoek zal bestaan uit een vergelijkende studie van verschillende dataformaten. Dit zal ondersteund worden door interviews gevoerd met verschillende belanghebbenden, waaronder huidige verstrekkers van baggergegevens, gerelateerde klanten en wetenschappelijke experts. Dit zal tot doel hebben om de redenen te achterhalen waarom deze individuen specifieke bestandsformaten gebruiken.

Het derde deel van het onderzoek zal zich richten op de implementatie van de data in het datasysteem of een proof-of-concept. Hierbij zal worden onderzocht of uniforme data effectief geïmplementeerd kan worden in het datasysteem, waarbij input en begeleiding nodig is van de verantwoordelijke partij voor het datasysteem.

%---------- Verwachte resultaten ----------------------------------------------
\section{Verwacht resultaat, conclusie}%
\label{sec:verwachte_resultaten}

Als eindresultaat zal er dus een uniform dataformaat zijn die voldoet aan FAIR- en open-standaarden en die efficiënt data kan importeren in het datasysteem. Dit dataformaat zal opgedrongen worden aan de verschillende leveranciers en contractanten van VLIZ. Er zal ook gekeken worden of er al een proof-of-concept mogelijk is op basis van dit resultaat.

Op basis van deze resultaten zal het mogelijk zijn om data van de menselijke activiteiten in de Schelde op te nemen in het dataportaal ScheldeMonitor.

